%% Согласно ГОСТ Р 7.0.11-2011:
%% 5.3.3 В заключении диссертации излагают итоги выполненного исследования, рекомендации, перспективы дальнейшей разработки темы.
%% 9.2.3 В заключении автореферата диссертации излагают итоги данного исследования, рекомендации и перспективы дальнейшей разработки темы.
\begin{enumerate}
    \item Предложена модель беспроводной централизованной сети связи при передаче видеоданных по протоколу HTTP. Найдена взаимосвязь между объективными характеристиками сети и качеством воспроизведения видео. Предложенная модель и найденная взаимосвязь позволяют осуществить аналитические исследования производительности беспроводных централизованных сетей при доминировании передачи видеоданных.
    \item Предложены алгоритмы вычисления граничных значений максимально возможной производительности беспроводных централизованных сетей для следующих критериев качества восприятия:
    \begin{itemize}
	    \item Нормированное отношение длительностей буферизации и просмотра при передаче неадаптивных видеопоследовательностей;
	    \item Отношение длительностей буферизации и просмотра с учетом средней битовой скорости видео при передаче адаптивного видео.
    \end{itemize}
    Полученные результаты могут быть использованы в качестве опорных значений при разработке новых алгоритмов планирования, учитывающие тип передаваемого трафика и требования к его обслуживанию, для существующих и последующих стандартов беспроводной связи.
    \item Предложен алгоритм планирования распределения ресурсов беспроводного канала, позволяющий минимизировать нормированное отношение длительностей буферизации и просмотра при передаче неадаптивного видео. Предложенный алгоритм позволяет увеличить емкость соты на $7-14\%$ в сравнении с известными решениями и демонстрирует производительность близкую к найденной аналитической границе.
\end{enumerate}
