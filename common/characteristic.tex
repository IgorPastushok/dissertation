{\actuality}
В настоящий момент времени огромной популярностью обладают сервисы хранения и передачи видео по протоколу прикладного уровня HyperText Transfer Protocol (HTTP). Данное явление вызвано множеством факторов, такими как бурное развитие мобильных устройств, увеличение аудитории социальных сетей и их плотная интеграция с сервисами хранения видеоконтента, рост популярности дистанционного обучения, видеокурсов, видеолекций и т. д. Подобная комбинация факторов приводит к доминированию передачи видеоданных в современных телекоммуникационных системах.

Важной особенностью передачи видео по протоколу HTTP является наличие двух технологий организации передачи видеоданных: неадаптивная (HTTP Progressive Download) и адаптивная (HTTP Adaptive Streaming), представленные в стандарте Dynamic Adaptive Streaming over HTTP (DASH). В современных реалиях пользователи обладают высокой мобильностью, что приводит к использованию беспроводных сетей связи в качестве носителя информации. Рост объемов видеотрафика приводит к экстремальным нагрузкам на беспроводную сеть, что проявляется в появлении эффектов деградации качества обслуживания абонентов: увеличение длительности ожидания начала воспроизведения и прерывание проигрывания видеоконтента при просмотре.

Общая производительность беспроводных централизованных сетей во многом определяется аспектами работы с беспроводным каналом связи, а именно алгоритмом распределения частотно-временных ресурсов между пользователями. В современных стандартах связи распределение ресурсов осуществляет алгоритм планирования (планировщик), установленный на канальном уровне базовой станции. Планировщики не регламентируются стандартами, и каждый производитель оборудования по собственному усмотрению выбирает принципы, в соответствии с которыми будет организовано распределение ресурсов радиоканала, что имеет непосредственное влияние на производительность системы в целом.

Таким образом, актуальной является задача исследования и разработки алгоритмов планирования для беспроводных централизованных сетей, которые обеспечивают высокую производительность и достаточный уровень качества восприятия при передаче видеоданных по протоколу HTTP.

%Большое количество усилий было направлено на изучение данной предметной области.
%В настоящей работе, для анализа производительности беспроводных сетей используется методы
При исследовании производительности алгоритмов планирования в беспроводных централизованных сетях для передачи видеоданных использовалась теория замкнутых систем массового обслуживания с конечным числом абонентов, проработанная A. Scherr, L. Kleinrock, В.М. Вишневским и А.И. Ляховым. Применение настоящей теории к исследованиям систем передачи видеоданных было представлено в ряде работ отечественных: Е.А. Бакин, Г.С. Евсеев, А.И. Парамонов, и зарубежных авторов: A. El Essaili, O. Oyman, V. Ramamurthi, исследующих производительность алгоритмов планирования для передачи видеоконтента. В большинстве подобных работ рассматривается неадаптивная технология передачи и предлагаются подходы для увеличения производительности в соответствии с рассматриваемым критерием качеством восприятия.

\aim\ настоящего диссертационного исследования является определение, вычисление и построение численных показателей максимально возможной производительности алгоритмов распределения ресурсов радиоканала при использовании адаптивной и неадаптивной технологии передачи видеоданных по протоколу HTTP, и разработка алгоритмов планирования распределения ресурсов беспроводного канала на базовой станции, производительность которых близка к максимально достижимой производительности.

Для~достижения поставленной цели необходимо решить следующие {\tasks}:
\begin{enumerate}
    \item Исследовать технологии адаптивной и неадаптивной передачи видеоданных по протоколу HTTP;
    \item Исследовать методы и критерии оценки качества восприятия видеоряда для адаптивной и неадаптивной передачи видеоданных и выделить факторы, обладающие наибольшим влиянием на качество восприятия;
    %\item Предложить критерии качества восприятия видеоданных для адаптивной и неадаптивной технологии передачи по протоколу HTTP;
    \item Ввести модель системы передачи видеоданных, включающую в себя модели компонентов системы передачи видео и беспроводной централизованной сети, и найти взаимосвязь между ее параметрами;
    \item Предложить аналитические оценки максимально возможной производительности введенной модели телекоммуникационной системы для исследованных критериев качества восприятия адаптивной и неадаптивной передачи видеоданных по протоколу HTTP;
    \item Разработать алгоритм планирования распределения ресурсов беспроводного канала связи на основе полученных аналитических результатов и продемонстрировать его производительность в сравнении с ними и существующими решениями.
\end{enumerate}

\textbf{Объект и предмет исследования}. Объектом исследования является беспроводная централизованная телекоммуникационная система с доминированием передачи видеоданных по протоколу HTTP.

Предмет исследования составляет алгоритм распределения частотно-временных ресурсов беспроводного канала связи на базовой станции при адаптивной потоковой передаче видеоданных по протоколу HTTP.

\textbf{Методы исследования.} При получении основных результатов работы использовались общие методы теории вероятностей и математической статистики, теории случайных процессов, методы математической оптимизации, в частности нелинейного и невыпуклого программирования, а также методы имитационного моделирования.

\novelty
\begin{enumerate}
    \item Построена трехкомпонентная модель системы передачи видеоданных по протоколу HTTP в беспроводных централизованных сетях связи, позволяющая провести аналитические исследования и сравнение производительности алгоритмов распределения ресурсов беспроводного канала;
    \item Найдена взаимосвязь между характеристиками модели системы передачи данных и проигрыванием видеоряда при передаче видео по HTTP протоколу;
    \item Предложен алгоритм вычисления нижней границы нормированного отношения длительностей буферизации и просмотра при неадаптивной передачи видеоданных по всевозможным алгоритмам планирования, удовлетворяющим введенной модели;
    \item Предложен и реализован алгоритм планирования, минимизирующий нормированное отношение длительностей буферизации и просмотра при неадаптивной передачи видеоданных;
    \item Найдена нижняя граница отношения длительностей буферизации и просмотра с учетом средней битовой скорости потока при адаптивной передаче видеоданных по всевозможным алгоритмам планирования и адаптации видеоряда, удовлетворяющим введенной модели.
\end{enumerate}

\influence\ диссертационной работы. Полученные в диссертационной работе результаты позволяют повысить производительность алгоритмов планирования распределения ресурсов беспроводного канала, и, как следствие, системы передачи данных в целом, для передачи видео по протоколу HTTP, что способствует увеличению емкости беспроводных систем. Также полученные результаты могут быть использованы для формирования требований к разрабатываемым стандартам связи текущих и последующих поколений.

\reliability. Результаты, полученные в диссертационной работе, согласуются с известными исследованиями передачи видеоданных по протоколу HTTP в беспроводных сетях. Основные результаты опубликованы в рецензируемых журналах и доложены на крупных международных конференциях.

\probation\ Основные результаты работы докладывались и обсуждались на следующих конференциях и симпозиумах в период с 2013 по 2017 гг.: на научных сессиях ГУАП; на конференции <<СПИСОК-2014>> на $15$-й конференции <<Conference of Open Innovations Association FRUCT>>; на $16$-м симпозиуме <<Problems of Redundancy in Information and Control Systems>>.

\textbf{Внедрение результатов.} Результаты работы были использованы в рамках проекта <<Разработка технических решений по построению внутриобъектовой сети мобильной радиосвязи>> ПАО <<Информационные телекоммуникационные технологии>>. Кроме того, результаты работы используются в учебном процессе кафедры инфокоммуникационных систем и кафедры безопасности информационных систем ГУАП.

\contribution\ Все результаты, представленные в тексте диссертационной работы, получены автором лично.

\publications\ Материалы, отражающие основное содержание и результаты диссертационной работы, опубликованы в $11$ печатных работах. Из них $2$ работы опубликованы в рецензируемых научных журналах, утвержденных в перечне ВАК, и $2$ работы опубликованы в изданиях, индексируемых в Scopus.

\defpositions
\begin{enumerate}
    \item Модель беспроводной централизованной системы связи при передаче видеоданных по протоколу HTTP, позволяющая производить аналитические исследования алгоритмов планирования распределения ресурсов радиоканала.
    \item Взаимосвязь характеристик беспроводной централизованной сети и воспроизведения видеоряда при передачи видео по протоколу HTTP.
    \item Алгоритм вычисления нижней границы нормированного отношения длительностей буферизации и просмотра для передачи неадаптивных видеопотоков.
    \item Алгоритм планирования распределения ресурсов радиоканала, обладающий наибольшей производительностью для критерия нормированного отношения длительностей буферизации и просмотра при передаче неадаптивных видеопотоков.
    \item Алгоритм вычисления нижней границы отношения длительностей буферизации и просмотра с учетом средней битовой скорости потока при адаптивной передачи видеоданных.
\end{enumerate}
