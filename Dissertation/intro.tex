\chapter*{Введение}
\addcontentsline{toc}{chapter}{Введение}
\markboth{}{}

\textbf{Актуальность темы.} 

В последнее время все большее распространение получает передача видеоданных от мобильных устройств по беспроводным сетям связи. Это связано как с существенным развитием технологий беспроводной передачи данных, так и с развитием мобильных устройств: беспроводных сенсоров, беспроводных камер видеонаблюдения, мобильных пользовательских устройств и т.д. В таких системах передатчик, как правило, характеризуется малыми вычислительными мощностями и ограниченной емкостью аккумуляторного устройства. При этом существенные ограничения накладываются как на мощность процессора, так и на объем памяти.

Существующие технологии сжатия видеоданных, в первую очередь подходы, описанные в стандартах серий ITU-T H.26x и ISO/IEC MPEG, не учитывают специфику источника информации в подобных системах, поэтому разработка новых методов сжатия визуальных данных, не требующих больших вычислительных затрат на стороне передатчика, является важной и актуальной задачей.

В диссертационной работе рассматривается распределенное кодирование видеоданных - подход к сжатию, основанный на методах кодирования зависимых источников, позволяющий существенно снизить сложность обработки на стороне передатчика. Решается \textbf{задача} эффективного восстановления промежуточных кадров на стороне декодера.

Различные аспекты кодирования зависимых источников (или распределенного кодирования источников) представлены в работах известных отечественных и зарубежных авторов (Г.Ш.Полтырев, С.И.Гельфанд, М.С.Пинскер, Д.Слепян, Д.Вулф, А.Вайнер). Однако, до недавнего времени, практическая реализация этих идей не была востребована. Только в конце 1990-х годов появились прикладные задачи, в которых применение распределенного кодирования могло дать преимущества по сравнению с существующими на тот момент подходами. В качестве перспективной области рассматривалось сжатие видеоданных в системах с ограниченными вычислительными ресурсами на стороне передатчика информации. Одним из существенных преимуществ от применения распределенного кодирования в данной системе, является то, что процедуру оценки и устранения временной избыточности, основанную на предсказании промежуточных кадров, можно выполнять на приемнике, за счет чего существенно снижается сложность обработки на передатчике. Процедуру межкадрового предсказания в таких системах принято называть \emph{генерацией дополнительной информации}. Кроме того, т.к. задача восстановления промежуточных кадров решается декодером, повышение степени сжатия возможно только за счет модификации приемника. В рамках данного класса прикладных задач были разработаны архитектуры кодеков, основанные на принципах распределенного кодирования источников. В последнее время большое внимание уделяется расширению и модификации этих архитектур с учетом появляющихся новых прикладных задач, таких как эффективное кодирование многомерных и многовидовых видеопоследовательностей. Несмотря на это, в базовой архитектуре остается ряд открытых вопросов. К их числу следует отнести учет сложного характера входных данных для процедуры межкадрового предсказания на стороне декодера, а также оценку параметров ошибок межкадрового предсказания. Также следует отметить, что в большинстве современных работ приводится описание эвристических подходов для решения практических задач, возникающих при распределенном кодировании визуальных данных. При этом теоретические модели данных исследованы не в полной мере, что не позволяет эффективно разрабатывать новые и улучшать существующие алгоритмы сжатия, использующие данный подход.

\textbf{Цель работы и задачи исследования}. Целью диссертационного исследования является повышение степени сжатия в видеокодеках, основанных на принципах распределенного кодирования источников, за счет улучшения существующих и разработки новых способов обработки данных на стороне декодера.

Для достижения цели исследования были поставлены следующие основные \textbf{задачи}.
\begin{enumerate}
\item Исследовать типовые методы сжатия видеоданных, основанные на принципах распределенного кодирования источников.
\item Исследовать способы формирования дополнительной информации на стороне декодера.
\item Предложить новый алгоритм генерации дополнительной информации, учитывающий особенности входных данных.
\item Исследовать статистические характеристики ошибок межкадрового предсказания, возникающих при генерации дополнительной информации.
\item Разработать модель ошибок предсказания промежуточных кадров на стороне декодера.
\item Предложить алгоритм оценки параметров ошибок межкадрового предсказания.
\end{enumerate}

\textbf{Объект и предмет исследования}. Объектом исследования является система сжатия видеоданных, основанная на принципах кодирования зависимых источников с дополнительной информацией на декодере.

Предмет исследования составляет процесс восстановления промежуточных кадров на стороне декодера.

\textbf{Методы исследования.} При получении основных результатов работы использовались общие методы системного анализа, методы теории вероятностей и математической статистики, теории случайных процессов, в частности марковских случайных полей, теории принятия решений, методы машинного зрения, а также методы имитационного моделирования.

\textbf{Научная новизна} диссертационной работы заключается в следующем.
\begin{enumerate}
\item Построена модель системы сжатия на базе распределенного кодирования для анализа алгоритмов восстановления промежуточных кадров на стороне декодера, позволяющая в одинаковых условиях производить сравнение различных методов сжатия видеоданных.
\item Предложен алгоритм генерации дополнительной информации декодера, отличающийся от существующих алгоритмов тем, что использует метод оценки движения, основанный на модели истинного движения объектов в видеоданных.
\item Предложен способ сравнительной оценки различных алгоритмов межкадрового предсказания в системах распределенного кодирования видеоданных, отличающийся от существующих тем, что позволяет в одинаковых условиях производить сравнительную оценку методов предсказания, не учитывая влияние прочих методов системы сжатия.
\item Впервые предложена модель ошибок предсказания промежуточных кадров, основанная на моделировании корреляционного шума с помощью скрытых Марковских сетей. 
\item Предложен алгоритм оценки параметров ошибок предсказания  промежуточных кадров, отличающийся от существующих тем, что позволяет в явном виде учитывать пространственную зависимость ошибок.
\end{enumerate}

\textbf{Практическая ценность} диссертационной работы. Полученные в диссертационной работе результаты позволяют повысить эффективность сжатия видеоданных в системах, основанных на распределенном кодировании источников, что способствует уменьшению энергопотребления и габаритных размеров сжимающего устройства.

\textbf{Степень достоверности.} Результаты, полученные в диссертационной работе, согласуются с известными теоретическими моделями. При проведении сравнительного анализа использовались эталонные реализации существующих методов. Основные результаты опубликованы в рецензируемых журналах и доложены на крупных международных конференциях.

\textbf{Апробация работы.} Основные результаты работы докладывались и обсуждались на следующих конференциях и симпозиумах в период с 2009 по 2014 гг.: на научных сессиях ГУАП; на $14$-ой конференции <<IEEE Multimedia Signal Processing>>; на $6$-ой конференции <<International Congress on Ultra Modern Telecommunications and Control Systems>>; на $13$-м симпозиуме <<Problems of Redundancy in Information and Control Systems>>; на $16$-ой конференции <<IEEE Multimedia Signal Processing>>.

\textbf{Внедрение результатов.} Результаты работы были использованы в рамках проекта <<Разработка цепочки фильтров постобработки видеоданных>>, осуществляемого ЗАО <<Интел А/О>>. Кроме того, теоретические результаты работы используются в учебном процессе кафедры инфокоммуникационных систем СПбГУАП.

\textbf{Публикации.} Материалы, отражающие основное содержание и результаты диссертационной работы, опубликованы в $17$ печатных работах. Из них 
$4$ работы опубликованы в рецензируемых научных журналах, утвержденных в перечне ВАК.

\textbf{Основные положения,} выносимые на защиту.
\begin{enumerate}
\item Алгоритм межкадрового предсказания на стороне декодера для систем распределенного кодирования видеоданных, основанный на процедуре временной интерполяции с учетом истинного движения объектов.
\item Модель ошибок предсказания промежуточного кадра для систем распределенного кодирования видеоданных, основанная на пространственном моделировании ошибок как Марковской сети.
\item Алгоритм оценки параметров ошибок в виртуальном канале, учитывающий пространственные характеристики корреляционного шума.
\item Метод сравнительной оценки алгоритмов межкадрового предсказания для систем распределенного кодирования видеоданных, основанный на моделировании кодека без обратной связи.
\end{enumerate}

\textbf{Структура и объем работы.} Диссертационная работа состоит из введения, четырех разделов, заключения и списка использованных источников. Работа содержит 155 страниц основного машинописного текста, 36 рисунков и 5 таблиц. Список использованной литературы содержит 89 наименований.