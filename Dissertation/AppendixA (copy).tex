\chapter{СЛУЧАЙНЫЕ МАРКОВСКИЕ ПОЛЯ}
\label{AppA}

Приведем описание основных понятий Марковских случайных полей (МСП)~\cite{Koller},~\cite{Li:2009:MRF:1529944},~\cite{petyushko2010o-markovskikh3421693}. Данный вероятностный аппарат широко используется в системах компьютерного зрения. МСП позволяют описывать вероятностные ансамбли с большим количеством зависимых случайных величин. При этом для МСП существует большое число алгоритмов поиска оптимальных конфигураций, обладающих доказанными свойствами сходимости. С точки зрения обработки изображений, МСП предоставляют удобный математический аппарат для моделирования сложных взаимодействий внутри изображения:
\begin{itemize}
\item изображения разбиваются на множество узлов, которые соответствуют отдельным пикселям или группам пикселей;
\item с каждым узлом ассоциируется скрытая переменная, которая позволяет в некотором смысле обосновать значение пикселя;
\item совместная вероятностная модель строится над узлами и скрытыми переменными;
\item статистические зависимости между скрытыми переменными выражаются за счет их группировки, выражающейся в наличии дуг между парами вершин графа вероятностной модели МСП.
\end{itemize}

Понятие МСП удобно формулировать в терминах задачи раскрашивания (или помечивания) элементов некоторого множества. Пусть задано множество из $m$ элементов. Далее будем называть эти элементы узлами. Обозначим множество индексов узлов как
\begin{equation*}
\mathcal{S}=\{1,2,...,m\}.
\end{equation*}
Узел, как правило, представляет собой точку или регион в Евклидовом 
пространстве. При обработке изображений в качестве узлов принято использовать 
пространственные координаты групп пикселей. Множество узлов может обладать 
свойством регулярности, например узлы, расположенные на двумерной решетке можно 
рассматривать как пространственно регулярное множество. Прямоугольную решетку 
для пикселей двумерного изображения, состоящего из $n$ строчек и $n$ столбцов 
можно ввести как 
\begin{equation*}
\mathcal{S} = \{ (i,j) \vert 1 \leq i,j \leq n \}.
\end{equation*}
Множество узлов, не обладающее регулярной структурой в пространстве, будем 
называть нерегулярным. Далее будем рассматривать только регулярные множества узлов.

Меткой узла будем называть случайное событие, которое может происходить в узле. Обозначим множество всех возможных меток для узла с индексом $i$ как $\mathcal{L}_i$. 

Задача помечивания узлов заключается в назначении каждому узлу c индексом $i \in \mathcal{S}$ некоторой метки из $\mathcal{L}_i$. Обозначим множество меток, назначенных всем узлам, как
\begin{equation*}
f = \{ f_1,f_2,...,f_m \}.
\end{equation*}
где $f_i \in \mathcal{L}_i$, $f \in \prod\limits_{i=1}^{m}\mathcal{L}_i$ (здесь 
под знаком произведения понимается декартово произведение множеств).
В терминах случайных полей множество $f$ принято называть конфигурацией 
случайного поля. 

Пусть над множеством $\mathcal{S}$ задано отношение соседства (система соседства). Множество соседей определяется как 
\begin{equation*}
\mathcal{N} = \{ \mathcal{N}_i| \forall i \in \mathcal{S} \}.
\end{equation*}
где $\mathcal{N}_i$~--~множество узлов, находящихся в отношении соседства с узлом $i$ (шаблон соседства). Отношение соседства обладает следующими свойствами:
\begin{enumerate}
\item Анти-рефлексивность: $i \notin \mathcal{N}_i$;
\item Симметричность: $i \in \mathcal{N}_{i'} \leftrightarrow i' \in \mathcal{N}_{i}$
\end{enumerate}
Для множества индексов $\mathcal{S}$ заданного над регулярным множеством узлов, множество соседей узла $i$ можно определить как множество узлов, находящихся на расстоянии не больше $r$ от $i$:
\begin{equation*}
\mathcal{N}_i = \{ i' \in \mathcal{S} \vert d(i,i') \leq r, i' \neq i \}.
\end{equation*}
где через $d(i,i')$ обозначено Евклидово расстояние между узлами с индексами $i$ и $i'$.

В том случае, если над элементами множества $\mathcal{S}$ задано отношение порядка, множество соседей можно определить в явном виде. Например, если $\mathcal{S}=\{1,2,...,m\}$ содержит индексы пикселей одномерного изображения, то множество соседей для некоторого внутреннего узла $i$ при $r=1$ можно определить как $\mathcal{N}_i=\{i-1,i+1\}$. Для пикселей находящихся на границе изображения, множества соседей будут включать по одному элементу: $\mathcal{N}_1=\{2\}$ и $\mathcal{N}_m=\{m-1\}$.

Пара $(\mathcal{S},\mathcal{N}) \triangleq G$ задает граф: $\mathcal{S}$ содержит множество вершин графа, $\mathcal{N}$ определяет дуги.

Введем понятие случайного Марковского поля. Обозначим через  $F=(F_1,F_2,...,F_m)$ многомерную случайную величину, причем $F_i$ принимает значение $f_i \in \mathcal{L}$. Величина $F$ называется случайным полем. Следует отметить, что приведенные далее рассуждения справедливы и в том случае, когда поле задается как множество случайных величин, а не как одна многомерная случайная величина. Обозначим через $F_i=f_i$ событие, заключающееся в том, что случайная величина $F_i$ принимает значение $f_i$, а через $(F_1=f_1,...,F_m=f_m)$ обозначим совместное событие. Совместное событие будем обозначать как $F=f$, где $f=(f_1,...,f_m)$~--~конфигурация $F$, соответствующая реализации поля.

Для дискретного множества $\mathcal{L}$ вероятность того, что $F_i$ примет значение $f_i$ обозначим как $p(f_i)$, а вероятность совместного события~--~$p(f)$.

\begin{definition}
    Случайное поле $F$ называется Марковским случайным полем над $\mathcal{S}$ по отношению к $\mathcal{N}$ если и только если выполняются два условия:
    \begin{itemize}
        \item положительность: $p(f)>0$, $\forall f \in \mathcal{L}^m$;
        \item марковость: $p(f_i \vert f_{\mathcal{S}\setminus\{i\}})=p(f_i|f_{\mathcal{N}_i})$;
    \end{itemize}
    где $\mathcal{S}\setminus\{i\}$~--~разность двух множеств (множество узлов из $\mathcal{S}$, кроме узла $i$), $f_{\mathcal{S}\setminus\{i\}}$~--~множество меток в узлах из $\mathcal{S}\setminus\{i\}$, и
    \begin{equation}
        f_{\mathcal{N}_i} = \{ f_i'\vert i' \in \mathcal{N}_i \}.
    \end{equation}
\end{definition}
Первое условие заключается в том, что не существует конфигурации поля с нулевой вероятностью. Второе свойство определяет локальные характеристики поля: только метки, соответствующие узлам, находящимся в отношении соседства, оказывают влияние (зависят) друг на друга.

Марковские случайные поля можно рассматривать как обобщение двухсторонних цепей Маркова изменив индексы времени на координаты пространственного положения.

Приведем один из ключевых результатов, известный как теорема Хаммерсли-Клиффорда, позволяющий рассчитывать вероятности различных конфигураций МСП.

Введем понятие клики. Кликой $c$ для графа $G$ называется полносвязный подграф графа $G$. Клика определяется как подмножество вершин множества $\mathcal{S}$. Она может состоять единственного узла $\{i\}$, пары соседних узлов $\{i,i'\}$, тройки соседних узлов $\{i,i',i''\}$ и т.~д. Обозначим множество клик, состоящих из одного узла как
\begin{equation*}
    \mathcal{C}_1 = \{ i \vert i \in \mathcal{S} \},
\end{equation*}
из двух узлов как
\begin{equation*}
    \mathcal{C}_2 = \{ \{i,i'\} \vert i' \in N_i, i \in \mathcal{S} \}
\end{equation*}
и т.~д. Тогда множество всех клик графа $G$ можно определить как
\begin{equation*}
    \mathcal{C} = \bigcup_i \mathcal{C}_i.
\end{equation*}

Будем называть потенциальной функцией $V_{c}(c)$, $c \in \mathcal{C}$, любую функцию, отображающую значения случайных величин из клики $c$ (конфигурацию клики) на вещественную ось.

\begin{definition}
    Дискретное распределение называется распределением Гиббса, если
    \begin{equation*}
        P(f) = \frac{1}{Z} e^{-\sum_{c \in \mathcal{C}}V_c(f_с)},
    \end{equation*}
    где $Z$~--~нормировочный коэффициент, обеспечивающий, что $P(f)$~--~распределение вероятностей, $f_с$~--~значения, которые приняли случайные величины, поставленные в соответствие с вершинами из клики $c$.
\end{definition}

\begin{theoremapp}
    Поле $F$ является Марковским случайным полем тогда и только тогда, когда $P(f)$ является распределением Гиббса.
\end{theoremapp}

Таким образом, чтобы задать распределение вероятностей над МСП, достаточно определить потенциальные функции на всех кликах графа.