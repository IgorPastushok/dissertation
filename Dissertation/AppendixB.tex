\chapter{ПАРАМЕТРЫ МОДЕЛИРОВАНИЯ}
\label{AppB}

\begin{longtable}[H]{| C{10cm} | C{7cm} |}
    \caption{Параметры моделирования}\label{tab:SimParams} \\
		\hline
 		\textbf{Название параметра} & \textbf{Значение} \\
 		\hline
 		\multicolumn{2}{|c|}{\textbf{Общие положения}} \\
		\hline
		Максимальная удаленность пользователей от базовой станции $X_{max}$ & 700 м \\
		\hline
		Типы трафика & Только видео \\
		\hline
		Длительность моделирования & 360000 с \\
		\hline
		Длительность интервала планирования & 1 мс \\
		\hline

		\multicolumn{2}{|c|}{\textbf{Беспроводной канал}} \\
		\hline
		Модель затухания распространения сигнала & Окамура-Хата \\
		\hline
		Окружение & Городская застройка \\
		\hline
		Тип города & Большой город \\
		\hline
		Мощность передатчика базовой станции & 44 дБм \\
		\hline
		Мощность передатчика пользовательского устройства & 23 дБм \\
		\hline
		Коэффициент усиления антенны базовой станции и пользовательского устройства & 10 дБ \\
		\hline
		Несущая частота & 2 ГГц \\
		\hline
		Ширина полосы передачи & 10 МГц \\
		\hline
		Модель ошибки при передаче данных & Отсутствует \\
		\hline

		\multicolumn{2}{|c|}{\textbf{Плоские замирания}} \\
		\hline
		Модель & Extended Pedestrian A (EPA) \\
		\hline
		Движение окружения  & 3 км/ч \\
		\hline
		Длительность генерируемой последовательности & 10 с \\
		\hline
		Несущая частота & 2 ГГц \\
		\hline

		\multicolumn{2}{|c|}{\textbf{TCP протокол}} \\
		\hline
		Стандарт & TCP NewReno \\
		\hline
		Размер фрагментации пакета  & 1440 Б \\
		\hline
		Минимальное значения таймаута & 1 с \\
		\hline
		Начальный размер окна  & 4320 Б \\
		\hline
		Ограничение начального роста окна быстрого старта & Неограничен \\
		\hline
		Размер буфера на приемной и передающей стороне  & 6 МБ \\
		\hline

		\multicolumn{2}{|c|}{\textbf{Алгоритм планирования для неадаптивных видеопотоков}} \\
		\hline
		Интервал усреднения оценки скорости передачи информации ($\bar{w}_{S}$) & 1000 мс \\
		\hline
		Интервал усреднения оценки максимальной пропускной способности канала ($\bar{w}_{C}$) & 1000 мс \\
		\hline
		Минимальная гарантированная скорость получения информации ($S^{min}$) & 100 Кб/с \\
		\hline
		Порог неактивности пользователя ($t^{act}$) & 4 с \\
		\hline

		\multicolumn{2}{|c|}{\textbf{Видеоконтент}} \\
		\hline
		Длительность видео & 300 с \\
		\hline
		Битовые скорости потока неадаптивных видеопоследовательностей & 1 Мбит/с \\
		\hline
		Битовые скорости потока адаптивных видеопоследовательностей & [1, 4.5] Мбит/с \\
		\hline

		\multicolumn{2}{|c|}{\textbf{Видеоплеер}} \\
		\hline
		Длительность сегмента видеоданных & 2 с \\
		\hline
		Размер начальной буферизации & 4 с \\
		\hline
		Максимальный размер буфера & 10 с \\
		\hline
		Адаптация видеопотока & Соответствует параметрам, представленным в таблице \ref{tab:DashJSParams} \\
		\hline

		\multicolumn{2}{|c|}{\textbf{Поведение пользователя}} \\
		\hline
		Начальная задержка при заказе видео & Равномерная случайная величина в отрезке [1, 60] с \\
		\hline
		Пауза между просмотрами видео & Усеченная экспоненциальная случайная величина в отрезке [15, 45] с со средним значением 30 с\\
		\hline
\end{longtable}